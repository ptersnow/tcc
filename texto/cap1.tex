\pagestyle{plain}

\begin{abstract}
O Problema da Mochila é um dos problemas de otimização combinatória mais estudados da classe NP-difícil. Este trabalho aborda uma variação do problema original, conhecida como Problema da Mochila Compartimentada, onde os itens são divididos em classes e a mochila pode ser dividida em compartimentos, de tal forma que cada compartimento armazene itens de uma mesma classe. O problema consiste de encontar a distribuição dos itens em compartimentos que maximize o valor de utilidade da mochila, obedecendo restrições quanto à capacidade máxima da mochila e de cada compartimento. O  principal resultado deste trabalho refere-se a apresentação de um algoritmo genético para o Problema da Mochila 0-1 e para o Problema da Mochila Compartimentada, incluindo divulgação dos resultados de experimentação e comparação destes algoritmos com outros que utilizam estratégicas convencionais para resolução dos mesmos problemas.
%Existem muitas abordagens para obter uma solução aproximada em tempo polinomial. Aqui nós consideraremos o clássico problema da mochila 0/1 e uma %variação desse problema, o problema da mochila compartimentada. No segundo problema, temos que determinar as capacidades adequadas de vários %compartimentos que podem vir a ser alocados em uma mochila e como esses compartimentos devem ser carregados.

\end{abstract}

\chapter*{Introdução} \label{sec:introducao}
O Problema da Mochila desperta muito interesse devido a sua vasta gama de aplicações e também pelo fato de que ele surge como subproblema de inúmeros outros problemas. Devido a sua importância, ele é amplamente estudado, possuindo diversas variantes agrupadas no que podem ser chamadas de Classes de Problemas da Mochila.

Ele se caracteriza, basicamente, pela escolha de um subconjunto de itens que irá otimizar um objetivo. Para tanto, cada item deve possuir um ``peso'' e um ``benefício'' e a mochila deve possuir uma capacidade máxima.  Deseja-se, então, escolher um subconjunto de itens que maximize o benefício da mochila, sem que a soma dos pesos dos itens selecionados ultrapasse a capacidade da mochila.

Neste projeto, além do clássico Problema da Mochila 0-1, mostramos uma variação do mesmo, chamado Problema da Mochila Compartimentada, onde a mochila pode ser dividida em compartimentos que, por sua vez, agrupam itens de uma mesma classe ou tipo. Esse problema foi proposto em 1996 e, embora não tenha uma solução polinomial, é de interesse que existam heurísticas ou algoritmos aproximativos para resolvê-lo rapidamente já que o problema possui várias aplicações práticas, incluindo a aplicação na indústria metalúrgica, que motivou sua proposta inicial.

Este trabalho está dividido da seguinte forma: no Capítulo~\ref{sec:mochila}, definimos e explicamos algumas das classes de Problemas da Mochila, bem como algumas técnicas tradicionais de projeto de algoritmos para resolução de alguns dos problemas apresentados. Este capítulo também apresenta os principais conceitos relacionados aos Algoritmos Genéticos. No Capítulo~\ref{sec:compartimentada}, definimos o Problema da Mochila Compartimentada e descrevemos uma forma de resolução do problema utilizando a Heurística da Decomposição. No mesmo capítulo, apresentamos uma simplificação do problema, e um possível algoritmo genético resolvê-lo.

No Capítulo~\ref{sec:resultados}, descrevemos todas as implementações e experimentações realizadas. Apresentamos três métodos de resolução do Problema da Mochila 0-1, um método utilizando força bruta para o Problema da Mochila Compartimentada e dois algoritmos genéticos, um para a resolução do Problema da Mochila 0-1 e o outro para o Problema da Mochila Compartimentada. Este capítulo inclui comparações relacionadas ao tempo de execução dos principais métodos e a comparação entre as soluções retornadas pelos algoritmos genéticos e pelos algoritmos exatos também implementados. Por fim, no capítulo~\ref{sec:conclusao}, apresentamos as considerações finais e o resumo dos principais resultados obtidos. 

